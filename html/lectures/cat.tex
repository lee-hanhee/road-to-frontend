\begin{summary}
    \begin{center}
        \begin{tabular}{ll}
            \textbf{Concept} & \textbf{Syntax} \\ 
            Opening and closing tags & \texttt{<tag>...</tag>} \\
            \multicolumn{2}{p{\linewidth}}{\begin{itemize}
                \item Text an element will display goes b/w its opening/closing tags. 
                \item e.g. h\# p, a, div, span, etc.
                \begin{itemize}
                    \item h\#: 1-6 to signify the importance of content below them in decreasing order.
                    \item a: Anchor tag to create hyperlinks (use href attribute).
                \end{itemize}
            \end{itemize}} \\
            Self-closing tags & \texttt{<tag />} \\
            Void elements & \texttt{<tag>} \\
            \multicolumn{2}{p{\linewidth}}{\begin{itemize}
                \item An element w/o a closing tag.
                \item e.g. img, br, hr, input, link, meta
            \end{itemize}} \\
            Attributes & \texttt{<tag attribute="value">} \\
            \multicolumn{2}{p{\linewidth}}{\begin{itemize}
                \item Special words inside the opening tag of an element to control the element's behavior.
                \item e.g. id, class, src, alt, href, style, etc.
                \begin{itemize}
                    \item src: Specifies the URL of an image.
                    \item alt: Specifies an alternate text for an image to improve accessibility and is displayed if the image cannot be loaded.
                    \item href: Specifies the URL of the page the link goes to.
                \end{itemize}
            \end{itemize}} \\
            Commnenting & \texttt{<!- - comment - ->} \\
            \multicolumn{2}{p{\linewidth}}{\begin{itemize}
                \item Comments to leave messages w/o affecting browser display 
                \item Can make code inactive.
            \end{itemize}} \\
            Content areas & \texttt{<header>, <footer>, <nav>, <main>, <article>, <section>, <aside>} \\
            \multicolumn{2}{p{\linewidth}}{\begin{itemize}
                \item Identify different content areas for modularity, Search Engine Optimization (SEO), and accessibility. 
                \item main: Represent main content of the body of a document. 
                \begin{itemize}
                    \item Content inside should be unique and not repeated elsewhere on the page.
                \end{itemize}
                \item Nesting: For readability, nest content by indenting.
            \end{itemize}} \\
        \end{tabular}
    \end{center}
\end{summary}
