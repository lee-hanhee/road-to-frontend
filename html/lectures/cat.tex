\begin{summary}
    \begin{center}
        \begin{tabular}{ll}
            \textbf{Concept} & \textbf{Syntax} \\ 
            Open/close tag elements & \texttt{<tag>...</tag>} \\
            \multicolumn{2}{p{\linewidth}}{\begin{itemize}
                \item Text an element will display goes b/w its opening/closing tags. 
                \item e.g. elements with o/c tags h\# p, a, div, span, ul etc.
                \begin{itemize}
                    \item Text
                    \begin{itemize}
                        \item h\#: 1-6 to signify the importance of content below them in decreasing order.
                        \item em: Emphasize text (italicize).
                        \item strong: Strongly emphasize text (bold).
                    \end{itemize}
                    \item List:
                    \begin{itemize}
                        \item ul: Unordered list to create a list of items w/o a specific order.
                        \item ol: Ordered list to create a list of items with a specific order.
                        \item li: List item to create an item in a list (ordered or unordered).
                    \end{itemize}
                    \item Form:
                    \begin{itemize}
                        \item form: Create a form to collect user input.
                        \item button: Create a clickable button (default behavior of a button is to submit a form to the location in the form's action attribute).
                        \item label: used to help associate the text for an input element with the input element itself (helpful for assistive technologies).
                        \item fieldset: Group related inputs and labels together in a form. (block-level element $\rightarrow$ appear on new line)
                        \item legend: Caption for the content in the fieldset element.
                    \end{itemize}
                    \item figcaption: Caption to describe the image contained within the figure element. 
                    \item a: Anchor tag to create hyperlinks (use href attribute) for text, images, etc.
                \end{itemize}
            \end{itemize}} \\
            Self-close tag elements & \texttt{<tag />} \\
            Void elements & \texttt{<tag>} \\
            \multicolumn{2}{p{\linewidth}}{\begin{itemize}
                \item An element w/o a closing tag.
                \item e.g. img, br, hr, input, link, meta
                \begin{itemize}
                    \item img: Embed an image in a document.
                    \item input: Collect data from a web form. 
                \end{itemize}
            \end{itemize}} \\
            Commenting & \texttt{<!- - comment - ->} \\
            \multicolumn{2}{p{\linewidth}}{\begin{itemize}
                \item Comments to leave messages w/o affecting browser display 
                \item Can make code inactive.
            \end{itemize}} \\
            Content area elements & \texttt{<header>, <footer>, <nav>, <main>, <article>, <section>, <aside>} \\
            & \texttt{</header>, </footer>, </nav>, </main>, </article>, </section>, </aside>} \\
            \multicolumn{2}{p{\linewidth}}{\begin{itemize}
                \item Identify different content areas for modularity, Search Engine Optimization (SEO), and accessibility. 
                \item Nesting: For readability, nest content by indenting.
                \item Examples:
                \begin{itemize}
                    \item main: Represent main content of the body of a document. 
                    \item section: Define sections in a document, such as chapters, headers, footers, or any other sections of the document.
                    \begin{itemize}
                        \item Content inside should be unique and not repeated elsewhere on the page.
                    \end{itemize}
                    \item figure: Represents self-contained content and will allow you to associate a caption with an image.
                \end{itemize}
            \end{itemize}} \\
        \end{tabular}
    \end{center}
\end{summary}
\newpage

\begin{summary}
    \begin{center}
        \begin{tabular}{ll}
            \textbf{Concept} & \textbf{Syntax} \\ 
            Attributes & \texttt{<tag attribute="value">} \\
            \multicolumn{2}{p{\linewidth}}{\begin{itemize}
                \item Special words inside the opening tag of an element to control the element's behavior.
                \item e.g. id, class, src, alt, href, style, target etc.
                \begin{itemize}
                    \item Image:
                    \begin{itemize}
                        \item src: Specifies the URL of an image.
                        \item alt: Specifies an alternate text for an image to improve accessibility and is displayed if the image cannot be loaded.
                    \end{itemize}
                    \item a: 
                    \begin{itemize}
                        \item href: Specifies the URL of the page the link goes to.
                        \item target: Specifies where to open the linked document (e.g. "\_blank" opens in a new tab).
                    \end{itemize}
                    \item Form:
                    \begin{itemize}
                        \item action: Where form data should be sent. 
                    \end{itemize}
                    \item Input: 
                    \begin{itemize}
                        \item type: Specifies the type of form element (e.g. text, password, submit, radio (circle), checkbox (square)).
                        \item name: For a form's data to be accessed by the location specified by action, you must give the text input a name attribute and assign it a value to represent the data being submitted.
                        \item value: Specify the data sent to the server when the form is submitted.
                        \begin{itemize}
                            \item  Radio: W/o value attribute, the default is "On" regardless of which button is clicked (use value to differentiate).
                        \end{itemize}
                        \item placeholder: Provides a hint to the user of what can be entered in the field.
                        \item required: Prevent a user from submitting a form without filling out a required field (no value needed).
                    \end{itemize}
                    \item Button:
                    \begin{itemize}
                        \item type: Specifies the type of button (e.g. submit, reset, button, checkbox).
                    \end{itemize}
                    \item General
                    \begin{itemize}
                        \item id: Used to identify specific HTML elements. This must be unique from all other id values for the entire page. 
                    \end{itemize}
                \end{itemize}
            \end{itemize}} \\
        \end{tabular}
    \end{center}
\end{summary}


